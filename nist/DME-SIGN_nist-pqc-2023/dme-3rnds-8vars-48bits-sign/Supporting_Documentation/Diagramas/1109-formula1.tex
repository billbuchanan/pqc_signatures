\documentclass[12pt]{article}
\usepackage{amsmath,amssymb}
\usepackage[utf8x]{inputenc}
\usepackage{amsfonts}
\usepackage[T1]{fontenc}
\usepackage{tikz}
\usepackage{pgf}
\usepackage{xcolor}
\usetikzlibrary{cd}


\begin{document}
\[
h:\{1,\dots,N\}\longrightarrow
\{11,\dots,1m,\dots,n1\dots,nm\}
\text{ tal que }
1m,\dots,nm\notin\operatorname{Im}(h)
\]

\[
H:\mathbb{F}_q^N\longrightarrow
\mathbb{F}_q^{nm}
\text{ tal que }
x_j=H(\underline{x})_{h(j)}
\]

\begin{gather*}
\tilde{L}_1=(L_{11},\dots,L_{1m}),\quad L_{1i}:\mathbb{F}_q^{m}\to \mathbb{F}_q^{m}\\
L_{1i}(\underline{x}_i)=\underline{x}_i A_{1i}\quad
A_{1i}\in\operatorname{M}_{m\times m}(\mathbb{F}_q),\quad 
\det A_{1i}\neq 0
\end{gather*}

\begin{gather*}
\tilde{\pi}_1=(\pi_1,\overset{n}{\dots},\pi_1),\quad \pi_1:\mathbb{F}_q^{m}\to\mathbb{F}_{q^{m}},\quad
\pi_1(u_1,\dots,u_m)=\alpha_1 u_1+\dots+\alpha_m u_m
\end{gather*}

\begin{gather*}
\tilde{\pi}_2=(\pi_2,\overset{m}{\dots},\pi_2),\quad \pi_2:\mathbb{F}_q^{n}\to\mathbb{F}_{q^{n}},\quad
\pi_1(v_1,\dots,v_n)=\beta_1 v_1+\dots+\beta_n v_n
\end{gather*}

Si suponemos $n\geq m$ la matriz $M$ se puede expresar matriciamente como

\[
\begin{matrix}
\underline{x}_1\\
\vdots\\
\underline{x}_m\\
\vdots\\
\underline{x}_n
\end{matrix}
\begin{pmatrix}
x_{11}&\dots&x_{1m}\\
\vdots&\ddots&\vdots\\
x_{n1}&\dots&x_{mm}\\
\vdots&\ddots&\vdots\\
x_{n1}&\dots&x_{nm}
\end{pmatrix}
\overset{M}{\longrightarrow}
\begin{matrix}
\underline{x}'_1\\
\vdots\\
\underline{x}'_m
\end{matrix}
\begin{pmatrix}
x_{11}&\dots&x_{1m}&x_{m+1,1}&\dots&x_{n,1}\\
\vdots&\ddots&\vdots&\vdots&\ddots&\vdots\\
x_{m1}&\dots&x_{mm}&x_{m+1,m}&\dots&x_{nm}
\end{pmatrix}
\]


\end{document}
