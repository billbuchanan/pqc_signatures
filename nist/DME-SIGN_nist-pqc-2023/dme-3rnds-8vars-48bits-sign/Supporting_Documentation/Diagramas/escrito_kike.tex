\documentclass[12pt,a4paper]{amsart}

% for computer reading only
%\usepackage[letterpaper,left=1.2in,right=1.2in,top=1in,bottom=1in]{geometry}

%\title{The prime divisors of the period and index of a Brauer class}
%\author{Benjamin Antieau and Ben Williams}
\usepackage{color}

\addtolength{\hoffset}{-1cm} \addtolength{\textwidth}{2cm}
\linespread{1.2}


%\usepackage{palatino, euler}
% \usepackage{setspace}
% \onehalfspacing

% AMSRefs is a big pile of garbage and has to go after euler and before xy and hyperref
%\usepackage{amsrefs}

\usepackage[pdfstartview=FitH]{hyperref}
\usepackage[english]{babel}
%\usepackage{geometry}
\usepackage[utf8x]{inputenc}
\usepackage{amsmath, amssymb, amsthm}
\usepackage{graphicx}
\usepackage{enumitem}
\usepackage{soul}
\usepackage[colorinlistoftodos]{todonotes}
\usepackage{xy}
% \usepackage{chngcntr}
\xyoption{all}
\usepackage{amsmath,amssymb}
\usepackage[utf8x]{inputenc}
\usepackage{amsfonts}
\usepackage[T1]{fontenc}
\usepackage{tikz}
\usepackage{pgf}
\usepackage{xcolor}
\usepackage{mathrsfs}
\usetikzlibrary{cd}


\newif\iftikz
\tikztrue

\iftikz
\usepackage{tikz}
\usetikzlibrary{%decorations.markings,
%matrix,
calc%,arrows.meta
}
\fi

\newtheorem{thm}{Theorem}[section]
\newtheorem{conj}[thm]{Conjecture}
\newtheorem*{yano}{Yano's Conjecture}
\newtheorem{prop}[thm]{Proposition}
\newtheorem{prop0}{Proposition}
\newtheorem{cor}[thm]{Corollary}
\newtheorem{lema}[thm]{Lemma}
\newtheorem{caso}{Case}
\newtheorem{prps}{Properties}
\newtheorem{clm}[thm]{Claim}
\theoremstyle{remark}
\newtheorem{remark}[thm]{Remark}
\theoremstyle{definition}
\newtheorem{dfn}[thm]{Definition}
\newtheorem{ntc}[thm]{Notation}
\newtheorem{ejm}[thm]{Example}
\newtheorem{paso}{Step}
%\numberwithin{equation}{section}
\DeclareMathOperator{\ord}{ord}






\DeclareMathOperator{\spec}{Spec}
\DeclareMathOperator{\supp}{Supp}
\DeclareMathOperator{\In}{In}
\DeclareMathOperator{\Gr}{Gr}
\newcommand\balpha{\boldsymbol{\alpha}}
\newcommand\bdx{{\mathbf x}}
\newcommand\bdy{{\mathbf y}}
\newcommand\bn{{\mathbb N}}
\newcommand\bz{{\mathbb Z}}
\newcommand\bq{{\mathbb Q}}
\newcommand\bc{{\mathbb C}}
\newcommand\br{{\mathbb R}}
\newcommand{\cD}{{\mathcal D}}
\newcommand{\cO}{{\mathcal O}}
\newcommand{\cC}{{\mathcal C R_\mu }}
\newcommand{\cA}{{\mathcal E}}
\DeclareMathOperator*{\res}{Res}

\newcommand\gfq{{\mathbb{F}_q}}





%    Absolute value notation
\newcommand{\abs}[1]{\lvert#1\rvert}
%    Blank box placeholder for figures (to avoid requiring any
%    particular graphics capabilities for printing this document).
\newcommand{\blankbox}[2]{%
  \parbox{\columnwidth}{\centering
%    Set fboxsep to 0 so that the actual size of the box will match the
%    given measurements more closely.
    \setlength{\fboxsep}{0pt}%
    \fbox{\raisebox{0pt}[#2]{\hspace{#1}}}%
  }%
}

\def\f{f^{-1}\{0\}}


\newcommand\enet[1]{\renewcommand\theenumi{#1}
\renewcommand\labelenumi{\theenumi}}



\title{Patente}



%    Information for first author
\author[I. Luengo]{I. Luengo${^1}$}
%    Address of record for the research reported here
\address{Departamento de \'Algebra, Universidad Complutense,
Plaza de las Ciencias s/n, Ciudad Universitaria, 28040 Madrid, SPAIN}
%    Current address
\curraddr{}
\email{iluengo@ucm.es}




\date{}


\begin{document} 

\maketitle


The $\mathbb{F}_q$-linear isomorphism $L_3=\ell\circ\tilde{L}_3\circ\pi_2^{-1}$
is defined as
\begin{center}
\begin{tikzcd}
(\mathbb{F}_{q^n})^{m}
\arrow[rrr,bend right=30,"L_3" below]
\arrow[r,,"{\pi}_2^{-1}","\sim" below]&(\mathbb{F}_{q}^n)^{m}
\arrow[r,"\tilde{L}_3" below]&(\mathbb{F}_{q}^n)^{m}
\arrow[r,"\ell^{-1}" below]&\mathbb{F}_{q}^{nm}
\end{tikzcd}

\end{center}
The morphism $\tilde{L}_3$ is defined as $\tilde{L}_3=(L_{31},\dots,L_{3n})$ where
$L_{3j}(x'_j)=x_j' A_{3j}$, $A_{3j}\in\mathscr{M}_{n\times n}(\gfq)$ and
$\det(A_{3j})\neq 0$.

The main part of the design of the system are the two exponential maps
$G_1$ and $G_2$ build with monomial maps as follows:
\[
G_1(x_1,\dots,x_m)=(x_1^{a_{11}}\cdot\ldots\cdot x_m^{a_{1m}},\dots,x_1^{a_{m1}}\cdot\ldots\cdot x_m^{a_{mm}}),\qquad 
G_1:(\mathbb{F}_{q^n})^m\to\mathbb{F}_{q^n})^m
\]
where $A_1=(a_{ij})\in\mathscr{M}_{m\times m}(\mathbb{Z}_{q^n-1})$ such that $d'_1=\det(A_1)$ is prime with $q^n-1$;
\[
G_2(x'_1,\dots,x'_n)=({x'_1}^{b_{11}}\cdot\ldots\cdot {x'_n}^{b_{1n}},\dots,{x'_1}^{b_{n1}}\cdot\ldots\cdot {x'_n}^{b_{nn}}),\qquad 
G_1:(\mathbb{F}_{q^m})^n\to\mathbb{F}_{q^m})^n
\]
where $B_2=(b_{ij})\in\mathscr{M}_{n\times n}(\mathbb{Z}_{q^m-1})$ such that $d'_2=\det(B_2)$ is prime with $q^m-1$

If $\underline{x}=(x_{11},\dots,x_{nm})\in\gfq^{nm}$ are the inicial coordinates, then
the composition of the five maps $L_1,G_1,L_2,G_2$ and $G_3$ allow us to compute 
the components of $F(\underline{x})$ as polynomials 
$F_i\in\mathbb{F}_q[x_{11},\dots,x_{nm}]$. In order to keep small the number
of monomials, we choose the matrices $A_1$ and $B_2$ with the following properties:

\begin{enumerate}
 \item The entries of $A_1$ and $B_2$ are of the form $p^a$.
 \item We fix two integers $s$ and $t$ such that the rows of $A_1$
 have at most $s$ non zero entries and the rows of $B_2$ have at most
 $t$ non zero entries. One can compute the monomials in the $F_i$
 with the algorithm described below, resulting that the total number
 of monomials is 
$MON=(b\cdot n^s)^t$ where $b$ depends on the mixing map~$M$.
\item The inverse maps $G_1^{-1}$ and $G_2^{-1}$ can be computed in the same
way from the inverse matrix of $A_1$ and $B_2$ respectively and
$F_1^{-1}$ is also polynomial.
\end{enumerate}
If the number of monomials in $F^{-1}$ is not very big, one can get the coefficient
of the polynomial by computing enough number of pairs $(x,F(x))$. To avoid
this attack we tak $A_1$ such that $d_1=\frac{1}{\det(A_1)}\mod{q^n-1}$
has a expansion in base~$p$ with $d_1=[K_0,\dots,K_\ell]$ with at least $s_1$
non vanishing~$K_i$ and the same with $B_2$ and $d_2=\frac{1}{\det(B_2)}$
(with at least $t_1$ non vanishing digits). The details of values of $t_1,s_1$
will be given when discussing the security of the system.

The public key of the system is $K_P=(h,\pi_0,F)$ and the private key is
given by $h$, $\pi_0$ and the five maps $L_1,\dots,L_3$ and their inverses
that can be used to encrypt and decrypt. Given an encrypted message $z=F(\underline{x})=
DM(\overline{x})$, one compute $\underline{x}=F^{-1}(z)$ and discard the randow entries
with the use of~$h$.

It is possible to get the monomials of the $F_i$ without computing the composition
of the five maps as follows: we start with $m$ lists that contain
the coordinates of the $\underline{x}_i$, $M_{01}=[x_{11},\dots,x_{1n}]$, \dots,
$M_{mn}=[x_{m1},\dots,x_{mn}]$, and we define the operations on lists: multiplication
and exponentiation. If $S=[s_1,\dots,s_m]$, $T=[t_1,\dots,t_m]$ then
$S\cdot T=[s_i\cdot t_i]$ and $S^a=[s_i^a]$.

With these notations, one can see that the exponential $G_1$ produce, on each component,
polynomials whose list of monomials is $N_{0k}=M_{01}^{a_{k1}}\cdot\ldots\cdot M_{0n}^{a_{kn}}$.

The mixing map $M$ determines that in the list of monomials of each $x'_k$ appears
the list $N_{0k}$, joint with the list $N_{0j}$ of the vectors that are placed
at the $m-n$ last entreis of $x'_k$. If $b_k$ is the number of vectors adjoined
to $x'_k$ then, if we denote by $P_{0k}$ ($k=1,\dots,n$) such list, then the final
list of monomials of each component after $G_2$ to each monomial ${x'_1}^{b_{k1}}\cdot\ldots\cdot{x'_n}^{b_{kn}}$ gives $Q_{0k}=P_{01}^{b_{k1}}\cdot\ldots\cdot P_{0n}^{b_{kn}}$.

Notice that when we apply the final $\gfq$-linear bijection $\tilde{L}_3$,
each component still have the same monomial, than means that there are $n$ groups
of $m$ polynomials $F_{k1},\dots,F_{km}$ such that they have the same monomials,
namely the list $Q_{0k}$.

It is clear that the number of monomials of $Q_{0k}$ is at most $((1+b_k)\cdot n^s)^t$.
So if we denote by $b_{\max}=\max_k(1+b_k)$, we get on each component
at most $(b_{\max}\cdot n^s)^t$ monomials.


\end{document}
